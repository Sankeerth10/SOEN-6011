\documentclass{article}
\usepackage[a4paper, total={6in, 8in}]{geometry}
\title{Sinh(x)}
\date{}
\begin{document}
\maketitle`
\section{Brief Description}
    Sinh(x) is short for Hyperbolic sine of the element x. The hyperbolic sine (and cosine) is a linear combination of two exponents of the Euler number, e. In mathematics, hyperbolic functions are analogues of the ordinary trigonometric functions, but defined for the unit hyperbola rather than on the unit circle.\par
    Hyperbolic functions occur in the calculations of angles and distances in hyperbolic geometry. They also occur in the solutions of many linear differential equations, cubic equations, and Laplace's equation in Cartesian coordinates.
    There are various ways to define Sinh(x). In terms of exponential functions it is defined as:
    \\
    \begin{displaymath}
       sinh(x) = \frac{e^{x} - e^{-x}}{2} 
    \end{displaymath}
    \\
    Where e is the Euler’s number (base for natural logarithms).
\section{Domain and Codomain}
The domain of sinh(x) is the set of Real numbers, ${\rm I\!R}$\\
The codomain of sinh(x) is the set of Real numbers, ${\rm I\!R}$
\section{Characteristics that make it unique}
\begin{itemize}
    \item Sinh(x) along with cosh x gives us all the points on the unit hyperbola i.e, $x^{2} - y^{2} = 1$ which in-turn gives rise to lot of hyperbolic trigonometric identities. These identities can be used for parametrizing and solving integrals.
    \item Tanh(x) defined as sinh(x)/cosh(x) describes the geometry of Special Theory of Relativity.
    \item The properties of the catenary are nicely described using the hyperbolic trigonometric functions. A catenary is the curve you get by hanging a chain of uniform density by its two endpoints. Flipping the catenary vertically will give you the ideal shape for an arch.
    \item If we rotate the unit hyperbola about the origin by 45 degrees in the anticlockwise direction, it will fit the equation xy = 1 or y = 1/x, giving us the remarkable curve from where the natural logarithm is derived.
\end{itemize}
\end{document}
