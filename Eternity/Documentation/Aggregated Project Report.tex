\documentclass{article}
\usepackage[a4paper, total={6in, 8in}]{geometry}
\usepackage[hidelinks]{hyperref}
\fontfamily{georgia}\selectfont
% Template-specific packages
\usepackage[utf8]{inputenc} % Required for inputting international characters
\usepackage[T1]{fontenc} % Output font encoding for international characters
\usepackage{mathpazo} % Use the Palatino font
\usepackage{geometry}
\usepackage{graphicx} % Required for including images
\usepackage{booktabs} % Required for better horizontal rules in tables
\usepackage{listings} % Required for insertion of code
\usepackage{algorithm}
\usepackage{algorithmic}
\usepackage{amsmath,algpseudocode}
\usepackage{enumerate} % To modify the enumerate environment
\usepackage[utf8]{inputenc}
\usepackage[english]{babel}
\usepackage{url}
\title{SOEN 6011 \\ Software Engineering Processes\\ Project Report \\ F3: sinh(x) \\ Supervised by: \\ Dr. Pankaj Kamthan}
\author{Sai Sankeerth Koduri (Id: 40195685)} 
\date{5th August,2022}
\begin{document}
\maketitle
\section{Problem 1}
\subsection{Brief Description}
    Sinh(x) is short for Hyperbolic sine of the element x. The hyperbolic sine (and cosine) is a linear combination of two exponents of the Euler number, e. In mathematics, hyperbolic functions are analogues of the ordinary trigonometric functions, but defined for the unit hyperbola rather than on the unit circle.\cite{1}\par
    Hyperbolic functions occur in the calculations of angles and distances in hyperbolic geometry. They also occur in the solutions of many linear differential equations, cubic equations, and Laplace's equation in Cartesian coordinates.\par
    Sinh(x) is defined as,
    \textbf{sinh($x$)} =   {(e\textsuperscript{x} - e\textsuperscript{-x})}{$\div$ 2},
    Where e is the Euler’s number (base for natural logarithms).
\subsection{Domain and Co-Domain}
The domain and co-domain of sinh(x) is the set of Real numbers, ${\rm I\!R}$ \cite{2}\newline
\subsection{Characteristics that make it unique}
\begin{itemize}
    \item Sinh(x) along with cosh x gives us all the points on the unit hyperbola i.e, $x^{2} - y^{2} = 1$ which in-turn gives rise to lot of hyperbolic trigonometric identities which can be used for parametrizing and solving integrals.\cite{3}
    \item Tanh(x) defined as sinh(x)/cosh(x) describes the geometry of Special Theory of Relativity.\cite{4}
    \item The properties of the catenary are nicely described using the hyperbolic trigonometric functions. A catenary is the curve you get by hanging a chain of uniform density by its two endpoints.
    \item item sinh($x$) $\approx$ cosh ($x$) for large x. Similarly, sinh($x$) $\approx$  -cosh ($x$) for large negative x.
    \item sinh($x$) is odd function , sinh($-x$) = - sinh($x$) and it has a period of 2$\pi$$\imath$
    
\end{itemize}
\subsection{Context of Use Model}
\begin{figure}[htp]
    \centering
    \includegraphics[width=12cm]{UseContextModel.png}
    \caption{Context of use Model}
    \label{Context of use Model}
\end{figure}
\section{Problem 2}
\subsection{Requirements}
\begin{enumerate}
\item Requirement 1 \newline
\textbf{ID: ET-F3-Sinh-REQ\_1}\newline
Type : Functional Requirement\newline
Description: The function should return a hyperbolic sine of input, if the user provides a real number as an input.\newline\newline\newline
\item Requirement 2 \newline
\textbf{ID: ET-F3-Sinh-REQ\_2}\newline
Type : Functional Requirement\newline
Description: The function should throw an error message stating that it is an invalid input, if the user provides an alphabet or a special character as an input.\newline
\item Requirement 3 \newline
\textbf{ID: ET-F3-Sinh-REQ\_3}\newline
Type : Functional Requirement\newline
Description: The calculated output value should be finite. If the value is not finite, it should throw an error message stating the output is not finite.\newline
\item Requirement 4 \newline
\textbf{ID: ET-F3-Sinh-REQ\_4}\newline
Type : Non Functional Requirement\newline
Description: The project should be designed in a way that it is modular and not tightly coupled.\newline
\end{enumerate}
\subsection{Assumptions}
\begin{enumerate}
    \item Assumption 1\newline
    \textbf{ID: ET-F3-Sinh-AS\_1} \newline
    Description: Even though the hyperbolic sine function has a domain of all possible real numbers, as the function will be implemented using java programming language, there is a limitation on user input because of the datatype.\newline
    The maximum input is + 1.79769313486231570E+308  and minimum input is - 1.79769313486231570E+308.
    \item Assumption 2\newline
    \textbf{ID: ET-F3-Sinh-AS\_2} \newline
    Description: Even if user inputs maximum allowed value,the output will be limited and output will be less than maximum allowed value and more than minimum allowed value.\newline
    The maximum input is + 1.79769313486231570E+308  and minimum input is - 1.79769313486231570E+308.
    \item Assumption 3\newline
    \textbf{ID: ET-F3-Sinh-AS\_3} \newline 
    Description: First 15 decimal points will be considered as a significant decimal point when the user will provide an input.
    \item Assumption 4\newline
    \textbf{ID: ET-F3-Sinh-AS\_4} \newline The value of e raised to x is calculated using the Taylor series,which goes upto infinity.Since there is limitation of datatype in java programming language infinite calculation is not possible.This is why we stop after the first 20 steps.
\end{enumerate}
\newpage
\section{Problem 3}
\subsubsection{Algorithm 1}
\newline
\textbf{Description:}
In this algorithm, the value of e(Euler's number) is not calculated explicitly. The value of e is declared as a constant and its value is copied from the e value in java.lang.Math library.\newline
This constant is then used to calculate the values of $e\textsuperscript{x}$ and $e\textsuperscript{-x}$ which are further used to obtain sinh(x). \newline
\textbf{Advantage:}
\begin{itemize}
    \item No explicit method call is required to calculate the value of $e$.
\end{itemize}
\textbf{Disadvantages:}
\begin{itemize}
    \item No way of increasing the accuracy of e as it is declared as a constant.
    \item Possibility of corruption of the value of e.
\end{itemize}
\newline
\newline
\textbf{Reasons to select this algorithm:}
\begin{itemize}
    \item No Recursion.
    \item Simple to implement.
\end{itemize}
\newline\newpage
\textbf{Pseudocode for Algorithm 1:}
\begin{algorithm}
\caption{sinh($x$) =   {($e\textsuperscript{x} - e\textsuperscript{-x}$)}{$\div$ 2}}
\begin{algorithmic} 
\REQUIRE $x \neq null$
\ENSURE $x \neq null$ and sinh($x$) =   {($e\textsuperscript{x} - e\textsuperscript{-x}$)}{$\div$  2}
\STATE $e \leftarrow $Declared as a constant
\STATE $e\textsuperscript{x} \leftarrow $GETERAISEDTOX($x$)
\STATE $e\textsuperscript{-x} \leftarrow $GETERAISEDTOX($-x$)
\STATE $sinhx \leftarrow (e\textsuperscript{x} - e\textsuperscript{-x})\div 2 $
\Statex
\Function{GETERAISEDTOX}{$x$}
\STATE $result \leftarrow e$
\newline
\IF{$x \textgreater $0}
\WHILE{$counter \textless x$}
     $result = result * $e\newline
     \STATE $counter \leftarrow $counter$+1$
\ENDFOR
\ENDWHILE\
\IF{$x \textless $0}
$y\leftarrow $-x\newline
\WHILE{$counter \textless y$}\newline
     $result = result * $e
     \STATE $counter \leftarrow $counter$+1$
\ENDWHILE
\ENDIF
\IF{$x = $0}
$result = 1$
\ENDIF
\State \Return $result$
\EndFunction
\end{algorithmic}
\end{algorithm}
\newline
\newline
\subsubsection{Algorithm 2} \newline\newline
\textbf{Description:}
\newline\normalfont 
In this algorithm,value of e is not calculated explicitly.The method GETERAISEDTOX calculates value of e raised to x using the Taylor series.\cite{5}
Since the algorithm should terminate at some point , as we can not calculate the Taylor series till infinity, we terminate the method when the counter value reaches 20.This gives us a very decent accuracy while calculating sinh(x).
\newline
\newline\textbf{Advantages:}
\begin{itemize}
    \item Accuracy can be increased by increasing the counter value where the function terminates
    \item e is not declared as a constant and there is no chance in corruption of the value which is likely in Algorithm 1.
\end{itemize}
\newpage
\newline\textbf{Disadvantages:}
\begin{itemize}
    \item Deciding the counter value where the method should terminate is difficult.
\end{itemize}
\newline\textbf{Reasons to select this algorithm:}
\begin{itemize}
    \item Flexibility in achieving an accurate result for sinh($x$) by increasing the the counter value at which the method will terminate.
    \item No chance in corruption of the value of e which is possible in Algorithm 1.
    \item Need no know the value of e beforehand, as we do not store it as a constant
\end{itemize}
\newline
\textbf{Pseudocode for Algorithm 2:}
\begin{algorithm}
\caption{Calculate sinh($x$) =   {($e\textsuperscript{x} - e\textsuperscript{-x}$)}{$\div$ 2}}
\begin{algorithmic} 
\REQUIRE $x \neq null$
\ENSURE $x \neq null$ and sinh($x$) =   {($e\textsuperscript{x} - e\textsuperscript{-x}$)}{$\div$  2}
\STATE $e\textsuperscript{x} \leftarrow $GETERAISEDTOX($x$)
\STATE $e\textsuperscript{-x} \leftarrow $GETERAISEDTOX($-x$)
\STATE $ sinhx \leftarrow (e\textsuperscript{x} - e\textsuperscript{-x})\div 2 $
\Statex
\Function{GETERAISEDTOX}{$x$} 
   \STATE $result \leftarrow 1$
\STATE $counter \leftarrow 20$
\WHILE{$counter \textgreater 0$}
     $result = 1 + ($x$ * $result$) $\div$ $counter;\newline
     \STATE $counter \leftarrow counter$-1
\ENDWHILE
    \State \Return $result$
\EndFunction
\end{algorithmic}
\end{algorithm}\newline
Considering the advantages and disadvantages of both the algorithms, Algorithm 2 is implemented.\newpage
\subsection{Mind map}
\begin{figure}[htp]
    \centering
    \includegraphics[width=12cm]{Mind map.png}
    \caption{Mind map for selecting the pseudo code format}
    \label{Mind map for selecting the pseudo code format}
\end{figure}
\section{Problem 4}
\subsection{Debugger}
A debugger is a computer program used by programmers to debug and test their programs. While programming, debugger helps to detect and resolve any bugs or unexpected behaviour.\cite{6}
During implementation of F3: sinh($x$) using java, I used IntelliJ debugger provided by the IntelliJ IDE.
\newline
\newline
\textbf{Advantages of IntelliJ Debugger}
\begin{itemize}
    \item IntelliJ debugger provides for an inline view while debugging.
    \item supports conditional breakpoints.
    \item Create breakpoints and watchpoints and also configure them.
    \item Pause and resume the debugger session at any time.
    \item Can debug Java applications remotely.
    \item shows the memory usage which can be used for analysis.
\end{itemize}
\begin{figure}[htp]
    \centering
    \includegraphics[width=12cm]{IntelliJ inline.png}
    \caption{Inline value view in IntelliJ debugger(Values in gray font)}
    \label{Inline value view in IntelliJ debugger(Values in gray font)}
\end{figure}\newline
\begin{figure}[htp]
    \centering
    \includegraphics[width=12cm]{Conditional Breakpoint.png}
    \caption{Support for conditional break points}
    \label{Support for conditional break points}
\end{figure}
\begin{figure}[htp]
    \centering
    \includegraphics[width=12cm]{Memory.png}
    \caption{Support for checking memory}
    \label{Support for checking memory}
\end{figure}
\newpage
\textbf{Disadvantages of IntelliJ Debugger}
\begin{itemize}
    \item It is not possible to use IntelliJ Debugger independently. 
    \item IntelliJ debugger requires high end configurations and consumes more memory compared to the other debuggers.
\end{itemize}
\subsection{Quality Attributes}
\begin{enumerate}
\item \textbf{Correctness}
\begin{itemize}
\item The correctness of the program is checked using the unit test cases.\cite{10}
\item With the help of these test cases, the expected value and the actual value are compared.
\item Code coverage of the test cases is 100\% to ensure the correctness and avoid unexpected behaviour
\end{itemize}
 \begin{figure}[htp]
    \centering
    \includegraphics[width=12cm]{Code Coverage.png}
    \caption{Code coverage for the Project}
    \label{Code coverage for the Project}
\end{figure}
\item \textbf{Efficiency}
\begin{itemize}
    \item No nested loops used.
    \item No recursion  used.
    \item Junit test suit (8 test cases) which executes all implemented functions takes  53 ms to complete.
     \begin{figure}[htp]
    \centering
    \includegraphics[width=12cm]{Passed Test Cases.png}
    \caption{ Junit test suit Execution time for all functions}
    \label{Execution time for all functions}
\end{figure}
\end{itemize}\newpage
\item \textbf{Robustness}\\
The implementation is robust because:
\begin{itemize}
\item Strict validation is carried out on the user input.
\item checking the input to see if it is compatible with the data types before any calculations. 
\item If the output is more than the range supported by the data types, an error message is thrown.
\end{itemize}
\item \textbf{Usability}
\newline Usability is achieved because:
\begin{itemize}
\item No complicated ui is used. A user-friendly command line ui is being used.
\item Error messages are informative and to the point.
\item Asks the users if want to continue or terminate the application.
\item The project can be exported as executable jar file.
\end{itemize}
\item \textbf{Maintainable}
\begin{itemize}
\item Maintainability is being achieved using a modular design.
\item
Informative comments are added.
\begin{figure}[htp]
\centering
\includegraphics[width=12cm]{Folder Structure.png}
\caption{Project Structure showing the modular design}
\label{Project Structure showing the modular design}
\end{figure}
\end{itemize}
\newpage
\item \textbf{Error Message and Exception Handling}
\begin{itemize}
\item
Responding to unexpected or undesirable events that occur while a computer program is running is known as exception handling. Without this procedure, exceptions would disrupt a program’s normal operation and lead to a crash. To avoid this, exception handling is used.
\item
In the Hyperbolic sine Function Calculator, if a user enters any other input other than real numbers(like alphabets or special characters), then
the user will be notified by an error message ”Please enter a valid input”
Moreover, if the user enters a number which is more than a java datatype can handle, it notifies with another error message” and for any general exception a message ”Fatal
error” is thrown by an exception handling mechanism
\end{itemize}
\newpage
\section{CheckStyle}
CheckStyle is a tool which helps programmers write Java code that adheres to a coding standard.\newline
Google CheckStyle\cite{9} is used in this project
\begin{figure}[htp]
\centering
\includegraphics[width=12cm]{Google checkstyle.png}
\caption{Google CheckStyle being used}
\label{Use of checkstyle in the project}
\end{figure}\newline
\textbf{Advantages}
\begin{itemize}
\item Easy to configure. It can be integrated into the build such that the project will build only if there are no checkStyle errors.
\item Makes the code more readable.
\end{itemize}
\textbf{Disadvantages}
\begin{itemize}
\item No check that finds redundant type casts.
\item Better alternatives such as sonarQube are available in the market.
\end{itemize}\newpage
\subsection{Programming style}
Style guides ensure consistency and predictability across the project\cite{7}.
I am using Google Programming style for java. 
Reasons to use Google Style guide :
\begin{itemize}
\item It makes code easier to read.
\item It allows programmer to focus on logic instead of making style decision
\item It makes file and variable name more predictable.
\item Programmer can import the style guide  in IDE and reformat code with the help of shortcuts.
\end{itemize}
\section{Problem 5}
\subsubsection{Junit Standards}
 With the help of increased presentation and structuring of the usage of JUnit and testing, one can increase understanding and appreciation of the overall value of testing in software development.\cite{8}
\end{enumerate}
\subsection{Junit traceability}
\subsection{Traceability to Requirements}
\begin{enumerate}
\item \textbf{ET-F3-Sinh-REQ\_1} \newline
Junit test case(s) : testValidInput(), testSinhX() \newline
Description : These tests verify if user provides a valid input, the function is showing a correct valid value.
\item
\textbf{ET-F3-Sinh-REQ\_2} \newline
Junit test case(s) : testInvalidInput()\newline
Description : This test verifies if user inputs invalid value (alphabet/special characters).If so, the function throws a proper error message or not
\item
\textbf{ET-F3-Sinh-REQ\_3} \newline
Junit test case(s) : testERaisedToX()\newline
Description : This test verifies obtained value of $e\textsuperscript{x}$ (Euler’s number) is finite or not.
\end{enumerate}
\begin{thebibliography}{9}
\bibitem{texbook}
Donald E. Knuth (1986) \emph{The \TeX{} Book}, Addison-Wesley Professional.
\bibitem{1}
\url{https://en.wikipedia.org/wiki/Hyperbolic_functions}
\bibitem{2}
\url{https://www.analyzemath.com/DomainRange/domain_range_functions.html.}
\bibitem{3}
\url{https://www.quora.com/What-is-sinh-x}
\bibitem{4}
\url{https://math.stackexchange.com/questions/1070254/what-is-the-importance-of-sinhx}
\bibitem{5}
\url{https://en.wikipedia.org/wiki/Taylor_series}
\bibitem{6}
\url{https://raygun.com/blog/java-debugging-tool}
\bibitem{7}
\url{https://www.codereadability.com/why-use-a-style-guide}
\bibitem{8}
\url{Michael Wick, Daniel E. Stevenson, and Paul J. Wagner. Using testing and junit across the curriculum, 02 2005.}
\bibitem{9}
\url{https://checkstyle.sourceforge.io/writingchecks.html.}
\bibitem{10}
\url{https://en.wikipedia.org/wiki/Debugger}
\bibitem{11}
\url{https://www.modernrequirements.com/blogs/how-to-make-your-requirement-ids-more-descriptive-and-distinctive-using-custom-id/}
\bibitem{12}
\url{https://standards.ieee.org/ieee/29148/6937/#:~:text=29148%2D2011,-ISO%2FIEC%2FIEEE&text=ISO%2FIEC%2FIEEE%2029148%3A,and%20their%20content%20are%20defined.}
\end{thebibliography}
\end{document}
